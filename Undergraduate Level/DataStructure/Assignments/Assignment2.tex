%------------------------------------------------------------------------------
% Template Documentation
%
% Instructions for Students:
% - Use this template to report your coding work for the assignment.
% - Include your name and student ID at the top of the document.
% - Copy the definitions of the classes and methods you implemented.
% - Provide a brief description of the implementation and how it meets the assignment requirements.
% - Ensure your code is well-commented and follows the provided structure.
% - Submit your completed report as instructed by your instructor.
%
% Example usage:
% \section*{Title and Student Information}
% \section*{Approach and Solution}
% \section*{Class Definitions}
% \section*{Method Implementations}
% \section*{Test Cases and Results}
% \section*{Discussion and Conclusion}
%
% Replace the placeholder text with your own content.
%------------------------------------------------------------------------------




\documentclass{style}



\title{\Large \bf
Assignment I: Library Management System}


\begin{document}

\noindent
\begin{tabular*}{\textwidth}{ p{3cm}  l}
\textbf{Student Name:} & \makebox[6cm]{\hrulefill} \\ % Write your name here
\textbf{Student ID:} & \makebox[6cm]{\hrulefill} \\ % Write your student ID here
\end{tabular*}

\begin{instructions}

    You are required to implement a Task Scheduler Implementation  using a queue data structure in C++.

    \begin{itemize}
        \item Implement a \textbf{Queue} class that manages tasks.
        \item Each task should be represented by a simulated execution time (e.g., an integer or float).
        \item Provide methods to add tasks to the queue, remove tasks from the queue, and display the current queue of tasks.
        \item Demonstrate the scheduling and processing of tasks in your report. Simulate the execution of at least 5 tasks by printing their execution times.
        \item Explain how the queue data structure is used to manage task scheduling.
    \end{itemize}

    \textbf{Submission Requirements:}
    \begin{itemize}
        \item Include your well-commented source code in the report.
        \item Provide sample test cases and their outputs.
        \item Briefly explain your design choices and how OOP principles and data structures are applied.
        \item Ensure your code is well-structured and follows best practices.
    \end{itemize}

\end{instructions}

\section{Introduction}
This document illustrates how to use the provided LaTeX class and environments for your programming assignment submissions. Use the \texttt{codelisting} environment for your code, and the \texttt{testcase} environment for worked examples.

\section{Sample Code Listing}
Below is an example of how to include C++ code in your report:

\begin{codelisting}
// Example: Hello World in C++
#include <iostream>
using namespace std;

int main() {
    cout << "Hello, world!" << endl;
    return 0;
}
\end{codelisting}

\section{Test Cases}
Here are some sample test cases you can include in your report:
\begin{testcase}[Swapping Two Numbers in C++]
\begin{codelisting}
int a = 5, b = 10;
swap(a, b);
cout << "a = " << a << ", b = " << b << endl;
\end{codelisting   }

Output:
\begin{codelisting}
a = 10, b = 5
\end{codelisting}
\end{testcase}

\section{Inserting Figures}
You can include figures in your report using the \texttt{figure} environment. Here is an example:   
\begin{figure}[h]
    \centering
    \includegraphics[width=0.5\textwidth]{ImageSample.png} % Replace with your image file
    \caption{Sample Image}
    \label{fig:sample-figure}
\end{figure}








\end{document}
