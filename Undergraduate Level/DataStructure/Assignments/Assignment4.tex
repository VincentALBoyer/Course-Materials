\documentclass{../LatexStyle/style}

\title{\Large \bf Assignment 4: Advanced Data Structures}

\begin{document}

\maketitle

\begin{instructions}
You are required to implement a min-heap in C++ using Object-Oriented Programming (OOP) principles. The min-heap will be used to manage a priority queue for task scheduling based on task priority.
Your program should:
    \begin{itemize}
    \item Implement a min-heap in C++ using a vector to manage a priority queue for task scheduling.
    \item Each task should have a priority (integer) and a description (string). The heap should always allow extraction of the task with the highest priority (lowest integer value).
    \item Your implementation must support insertion of new tasks, extraction (pop) of the highest-priority task, and heapify operations to maintain the heap property.
    \item Write a main program that demonstrates scheduling tasks by priority, showing the order in which tasks are executed.
    \item Include comments in your code to explain key sections and logic.
    \item Provide at least two test cases that show your heap correctly schedules and executes tasks based on priority.
\end{itemize}
    \textbf{Submission Requirements:}
    \begin{itemize}
        \item Include your well-commented source code in the report.
        \item Provide sample test cases and their outputs.
        \item Briefly explain your design choices and how OOP principles are applied.
        \item Ensure your code is well-structured and follows best practices.
    \end{itemize}
\end{instructions}





\section{Sample Code Listing}
Below is an example of how to include C++ code in your report:

\begin{codelisting}
// Example: Hello World in C++
#include <iostream>
using namespace std;

int main() {
    cout << "Hello, world!" << endl;
    return 0;
}
\end{codelisting}

\section{Test Cases}
Here are some sample test cases you can include in your report:
\begin{testcase}[Swapping Two Numbers in C++]
\begin{codelisting}
int a = 5, b = 10;
swap(a, b);
cout << "a = " << a << ", b = " << b << endl;
\end{codelisting   }

Output:
\begin{codelisting}
a = 10, b = 5
\end{codelisting}
\end{testcase}

\section{Inserting Figures}
You can include figures in your report using the \texttt{figure} environment. Here is an example:   
\begin{figure}[h]
    \centering
    \includegraphics[width=0.5\textwidth]{ImageSample.png} % Replace with your image file
    \caption{Sample Image}
    \label{fig:sample-figure}
\end{figure}
\end{document}
